\documentclass[11pt]{beamer}
% => essayer différents thèmes (en décommantant une des trois lignes suivantes)
%\usetheme{PaloAlto}
%\usetheme{Madrid}
\usetheme{Copenhagen}
%\usetheme{Marburg}

% => il est possible, pour un thème donné, de modifier seulement la couleur
%\usecolortheme{crane}
%\usecolortheme{seahorse}

%\useoutertheme[left]{sidebar}

\usepackage[utf8]{inputenc}
\usepackage[T1]{fontenc}
\usepackage[english]{babel}
\usepackage{graphicx}
\newenvironment<>{examplefirst}[1]{%
  \setbeamercolor{block title}{fg=white,bg=red!75!black}%
  \begin{block}#2{#1}}{\end{block}}
\newenvironment<>{examplesecond}[1]{%
  \setbeamercolor{block title}{fg=white,bg=blue!75!black}%
  \begin{block}#2{#1}}{\end{block}}
  
\author{Mourad Daoudi}
\title{Les Réseaux de Pétri}
%\subtitle{}
%\logo{}
\institute{USTHB}
\date{Jeudi 25 Juin}
%\subject{}
%\setbeamercovered{transparent}
%\setbeamertemplate{navigation symbols}{}

\begin{document}
	\maketitle
	\begin{frame}[shrink=15]
		\frametitle{Sommaire}
		\tableofcontents{}
	\end{frame}





	\section{Introduction}
		\begin{frame}
		\frametitle{Définition génerale}
	
	
		\begin{examplefirst} {Rappel d'histoire}
		Les réseaux de Petri ont été inventés par le mathématicien allemand Carl Alain Petri dans
			les années 1960.
		\end{examplefirst}
	
	
		
		
		\end{frame}
	
	\section{Notations et règles de franchissement}

	\subsection{ Places, Transitions et Arcs       }
		\begin{frame}
	\frametitle{Définitions}
	\begin{examplesecond} {Un réseau de pétri c'est quoi ?}
		\setbeamercovered{invisible}
	   \begin{itemize}
	   \item  \uncover<.1-> { un graphe}
	   \item \uncover<.2-> {formé de deux types de nœuds appelés places et transitions, reliés par des arcs orientés}
	   \item \uncover<.3->  {biparti, c.-à-d. qu’un arc relie alternativement une place à une transition et une transition à une place}
	   \end{itemize}
	   
	   
	\end{examplesecond}
	\begin{examplefirst} {Remarques}
			\setbeamercovered{invisible}
			
			   \begin{itemize}
				   \item  \uncover<.4-> {Une place \textbf{(pi )} modélise les ressources utilisées dans le système.
				   }
				   \item \uncover<.5 -> { Une transition \textbf{(ti )} modélise les actions sur les ressources.}
		
				   \end{itemize}
	\end{examplefirst}
		
	\end{frame}
	\begin{frame}
	\frametitle{Example}
	\centering
	\begin{center}
		\includegraphics<1>[scale=0.40]{petriPlaces.png}
	\end{center}
	
	\end{frame}
	\subsection{ Marquages                }
		\begin{frame}
	\frametitle{}
	\end{frame}
	\subsection{ Franchissement             }
		\begin{frame}
	\frametitle{}
	\end{frame}
	\subsection{ Réseaux particuliers          }
		\begin{frame}
	\frametitle{}
	\end{frame}
	\section{Propriétés des réseaux de Petri}
		\begin{frame}
	\frametitle{}
	\end{frame}
	\subsection{ Graphe de Marquage Accessible (GMA)            }
		\begin{frame}
	\frametitle{}
	\end{frame}
	\subsection{ Le vecteur d’occurrence et l’équation de changement d’état }
		\begin{frame}
	\frametitle{}
	\end{frame}
	\subsection{ Quelques propriétés qualitatives                 }
		\begin{frame}
	\frametitle{}
	\end{frame}
	\subsubsection{ Bornitude                         }
		\begin{frame}
	\frametitle{}
	\end{frame}
	\subsubsection{ Pseudo-vivacité                      }
		\begin{frame}
	\frametitle{}
	\end{frame}
	\subsubsection{ Quasi-vivacité                       }
		\begin{frame}
	\frametitle{}
	\end{frame}
	\subsubsection{ Vivacté                          }
		\begin{frame}
	\frametitle{}
	\end{frame}
	\subsubsection{ Réseau sans blocage                   }
		\begin{frame}
	\frametitle{}
	\end{frame}
	\subsubsection{ Etat d’accueil                       }
		\begin{frame}
	\frametitle{}
	\end{frame}
	\subsubsection{ Conservation                       }
		\begin{frame}
	\frametitle{}
	\end{frame}
	\subsection{ Types de réseaux de Petri                    }
		\begin{frame}
	\frametitle{}
	\end{frame}
	\subsubsection{ Réseaux de Petri généralisés               }
		\begin{frame}
	\frametitle{}
	\end{frame}
	\subsubsection{ Réseaux de Petri à capacités              }
		\begin{frame}
	\frametitle{}
	\end{frame}
	\subsubsection{ Graphe de marquage                   }
		\begin{frame}
	\frametitle{}
	\end{frame}
	\subsection{ Arborescence de couverture                   }
		\begin{frame}
	\frametitle{}
	\end{frame}
	\subsubsection{ Algorithme de contstruction d’un graphe de marquage}
		\begin{frame}
	\frametitle{}
	\end{frame}
\end{document}