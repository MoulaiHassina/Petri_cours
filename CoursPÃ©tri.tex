\documentclass[11pt]{beamer}
\usetheme{Marburg}

\usepackage[utf8]{inputenc}
\usepackage[T1]{fontenc}
\usepackage[english]{babel}
\usepackage{graphicx}

\author{Mourad Daoudi}
\title{Les Réseaux de Pétri}
%\subtitle{}
%\logo{}
\institute{USTHB}
\date{Jeudi 25 Juin}
%\subject{}
%\setbeamercovered{transparent}
%\setbeamertemplate{navigation symbols}{}

\begin{document}
	\maketitle
	\begin{frame}[shrink=15]
		\frametitle{Sommaire}
		\tableofcontents{}
	\end{frame}

	
	\section{Notations et règles de franchissement}
		\begin{frame}
	\frametitle{}
	\end{frame}
	\subsection{ Places, Transitions et Arcs       }
		\begin{frame}
	\frametitle{}
	\end{frame}
	\subsection{ Marquages                }
		\begin{frame}
	\frametitle{}
	\end{frame}
	\subsection{ Franchissement             }
		\begin{frame}
	\frametitle{}
	\end{frame}
	\subsection{ Réseaux particuliers          }
		\begin{frame}
	\frametitle{}
	\end{frame}
	\section{Propriétés des réseaux de Petri}
		\begin{frame}
	\frametitle{}
	\end{frame}
	\subsection{ Graphe de Marquage Accessible (GMA)            }
		\begin{frame}
	\frametitle{}
	\end{frame}
	\subsection{ Le vecteur d’occurrence et l’équation de changement d’état }
		\begin{frame}
	\frametitle{}
	\end{frame}
	\subsection{ Quelques propriétés qualitatives                 }
		\begin{frame}
	\frametitle{}
	\end{frame}
	\subsubsection{ Bornitude                         }
		\begin{frame}
	\frametitle{}
	\end{frame}
	\subsubsection{ Pseudo-vivacité                      }
		\begin{frame}
	\frametitle{}
	\end{frame}
	\subsubsection{ Quasi-vivacité                       }
		\begin{frame}
	\frametitle{}
	\end{frame}
	\subsubsection{ Vivacté                          }
		\begin{frame}
	\frametitle{}
	\end{frame}
	\subsubsection{ Réseau sans blocage                   }
		\begin{frame}
	\frametitle{}
	\end{frame}
	\subsubsection{ Etat d’accueil                       }
		\begin{frame}
	\frametitle{}
	\end{frame}
	\subsubsection{ Conservation                       }
		\begin{frame}
	\frametitle{}
	\end{frame}
	\subsection{ Types de réseaux de Petri                    }
		\begin{frame}
	\frametitle{}
	\end{frame}
	\subsubsection{ Réseaux de Petri généralisés               }
		\begin{frame}
	\frametitle{}
	\end{frame}
	\subsubsection{ Réseaux de Petri à capacités              }
		\begin{frame}
	\frametitle{}
	\end{frame}
	\subsubsection{ Graphe de marquage                   }
		\begin{frame}
	\frametitle{}
	\end{frame}
	\subsection{ Arborescence de couverture                   }
		\begin{frame}
	\frametitle{}
	\end{frame}
	\subsubsection{ Algorithme de contstruction d’un graphe de marquage}
		\begin{frame}
	\frametitle{}
	\end{frame}
\end{document}