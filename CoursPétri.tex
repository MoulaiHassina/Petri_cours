\documentclass[11pt]{beamer}
% => essayer différents thèmes (en décommantant une des trois lignes suivantes)
%\usetheme{PaloAlto}
\usetheme{Madrid}
%\usetheme{Copenhagen}
%\usetheme{Marburg}

% => il est possible, pour un thème donné, de modifier seulement la couleur
%\usecolortheme{crane}
\usecolortheme{seahorse}
\setbeamertemplate{background canvas}{\includegraphics[width=\paperwidth,height=\paperheight]{page.jpg}}

\useoutertheme[left]{sidebar}

\usepackage[utf8]{inputenc}
\usepackage[T1]{fontenc}
\usepackage[english]{babel}
\usepackage{graphicx}
\newenvironment<>{examplefirst}[1]{%
  \setbeamercolor{block title}{fg=white,bg=red!75!black}%
  \begin{block}#2{#1}}{\end{block}}
\newenvironment<>{examplesecond}[1]{%
  \setbeamercolor{block title}{fg=white,bg=blue!75!black}%
  \begin{block}#2{#1}}{\end{block}}
  
\author{Mourad Daoudi}
\title{Les Réseaux de Pétri}
%\subtitle{}
%\logo{}
\institute{USTHB}
\date{Jeudi 25 Juin}
%\subject{}
%\setbeamercovered{transparent}
%\setbeamertemplate{navigation symbols}{}

\begin{document}
	\maketitle
	\begin{frame}[shrink=15]
		\frametitle{Sommaire}
		\tableofcontents{}
	\end{frame}
	\AtBeginSection[]{
		
		\begin{frame}
				\frametitle{Sommaire}
				\tableofcontents[currentsection]
			\end{frame}
		
		}
		\AtBeginSubsection[]{
			
			\begin{frame}
				\frametitle{Sommaire}
				\tableofcontents[currentsubsection]
			\end{frame}
			}





	\section{Introduction}
		\begin{frame}
		\frametitle{Définition génerale}
	
	
		\begin{examplefirst} {Rappel d'histoire}
		Les réseaux de Petri ont été inventés par le mathématicien allemand Carl Alain Petri dans
			les années 1960.
		\end{examplefirst}
	
	
		
		
		\end{frame}
	
	\section{Notations et règles de franchissement}

	\subsection{ Places, Transitions et Arcs       }
		\begin{frame}
	\frametitle{Définitions}
	\begin{examplesecond} {Un réseau de pétri c'est quoi ?}
		\setbeamercovered{invisible}
	   \begin{itemize}
	   \item  \uncover<.1-> { un graphe}
	   \item \uncover<.2-> {formé de deux types de nœuds appelés places et transitions, reliés par des arcs orientés}
	   \item \uncover<.3->  {biparti, c.-à-d. qu’un arc relie alternativement une place à une transition et une transition à une place}
	   \end{itemize}
	   
	   
	\end{examplesecond}
	
	\setbeamercovered{transparent}
	\begin{examplefirst} {Remarques}
		\setbeamercovered{transparent}
			
			   \begin{itemize}
				   \item  \uncover<.4->{Une place \textbf{(pi )} modélise les ressources utilisées dans le système.}
				   \item \uncover<.5 ->{ Une transition \textbf{(ti )} modélise les actions sur les ressources.}
		
				   \end{itemize}
	\end{examplefirst}
		
	\end{frame}
	\begin{frame}
	\frametitle{Exemples}
	\begin{block}{Exemples}
	la place p1 est en entrée de la transition t1 et p2 est en sortie de t1 .
	\end{block}
	\centering
	\begin{center}
		\includegraphics<1>[scale=0.40]{petriPlaces.png}
	\end{center}
	
	
	\end{frame}
	\begin{frame}

	 	\begin{block}{Remarques}
	 	-Une transition sans place en entrée est une transition source.
	 	
	 	-Une transition sans place en sortie est une transition puits.
	 	\end{block}
	 	\centering
	 	\begin{center}
	 		\includegraphics<1>[scale=0.40]{PuitSource.png}
	 	\end{center}
	\end{frame}
	\subsection{ Marquages }
		\begin{frame} [shrink=15]
	\frametitle{Marquage}
	\begin{examplefirst} {Le Marquage}
	Chaque place \textbf{\textit{(pi )}} d’un RdP peut contenir un ou plusieurs marqueurs (jetons). 
	
	La configuration complète du réseau, avec toutes les marques positionnées, forme le marquage et définit
	l’état du réseau (et donc l’état du système modélisé).

	
	\end{examplefirst}
		\centering
		 	\begin{center}
		 		\includegraphics<1>[scale=0.40]{marquage.png}
		 	\end{center}
		 	\begin{itemize}
		 		\item P1 ,P2,P3 sont des places .
		 		\item T1 est une transition qui permet de passer de P1 vers Deux places P2 et P3 .
		 		\end{itemize}
	\end{frame}
	\subsection{ Franchissement             }
		\begin{frame}
	\frametitle{Franchissement}
	\textbf{Franchissement}
	
	C’est le formalisme qui permet de passer d’un marquage à un autre, ce qui rend compte de
	l’évolution du système modélisé. Une transition est franchissable si chacune des places en
	entrée compte au moins un jeton ; dans ce cas :
	\setbeamercovered{invisible}
	\begin{enumerate}
	\item \uncover<.1->{le franchissement est une opération indivisible (atomique)}
	\item \uncover<.2->{ un jeton est consommé dans chaque place en entrée}
	\item \uncover<.3->{ un jeton est produit dans chaque place en sortie}
	\end{enumerate}
	

	\end{frame}
		\begin{frame}{Exemples de franchissement}
	Voici des exemples de franchissement avec deux réseaux différents.

		
	
				\centering
					 	\begin{center}
					 		\includegraphics<1>[scale=0.32]{franchissement1.png}
					 	\end{center}
					 	
					 	\centering
					 		\begin{center}
					 				\includegraphics<1>[scale=0.32]{franchissement2.png}
					 		\end{center}
					 	
					 	
				
		\end{frame}
	\subsection{ Réseaux particuliers }
		\begin{frame}
		Il existe des réseaux particuliers on va dans la suite de ce cours  citer quelques uns .
	\frametitle{Graphe d'état}
	\begin{examplesecond}{Graphe d'état}
	 un graphe d'état a une particularité qui est relative à ses transitions tel que , chaque transition ne dispose que d’une place en entrée et une place en sortie.
	
	
	\end{examplesecond}
	
	\centering
						 	\begin{center}
						 		\includegraphics<1>[scale=0.32]{grapheetat.png}
						 	\end{center}
	
	
	\end{frame}
		\begin{frame}
		\frametitle{Réseau sans conflit}
			\begin{examplesecond}{Réseau sans conflit}
			   Un réseau sans conflit est un réseau où chaque place n’a qu’une transition en sortie.	
			\end{examplesecond}
				\centering
				\begin{center}
				\includegraphics<1>[scale=0.32]{sansconflit.png}
				\end{center}
    		
		\end{frame}
		
    	
    	\begin{frame}
		\frametitle{Réseau simple}
		\begin{examplesecond}{Réseau simple}
		Les réseaux dits simples sont des  réseaux avec conflit(s) où chaque transition n’intervient au	plus que dans une situation de conflit.
		\end{examplesecond}
				\centering
				\begin{center}
				\includegraphics<1>[scale=0.32]{simpleR.png}
				\end{center}
		\end{frame}
			

    	\begin{frame}
		\frametitle{Les Graphes purs}
		\begin{examplesecond}{Graphe pur}
		Les  Graphes purs sont ceux dont aucune place n'est à la fois en entrée ou en sortie de la même transition.
		\end{examplesecond}
				\centering
				\begin{center}
				\includegraphics<1>[scale=0.32]{graphepur.png}
				\end{center}
		\end{frame}
	\begin{frame}
		\frametitle{Définitions }
		\begin{examplesecond}{Définition}
			 Un réseau de Petri est défini par le tuple \textbf{\textit{(P, T, Pré, Post, $M_{0}$ )}}
						
						\begin{itemize}
					\item 	\textbf{ P }: ensemble de places $p_{i}$
						\item  \textbf{ T }: ensemble de transitions
						\item \textbf{ Pré} : \textit{Pré(p, t)} est une valeur ($≥$ 0) associée à l'arc allant de la place p à la transition t
						\item   \textbf{Post} : \textit{Post(p, t)} est une valeur ($≥$ 0) associée à l'arc allant de la transition \textit{t} à la place \textit{p}
						\item  $M_{0}$ : vecteur décrivant le marquage initial, $M_{0}$  = ($M_{0}$  ($p_{1}$ ), . . . , $M_{0}$  ($p_{n}$)).
							 nombre de jetons dans la place $p_{1}$
							\end{itemize}
			\end{examplesecond}				
		\end{frame}

	
	\section{Propriétés des réseaux de Petri}
		\begin{frame}
	\frametitle{}
	\end{frame}
	\subsection{ Graphe de Marquage Accessible (GMA)}
		\begin{frame}
	\frametitle{Graphe de Marquage Accessible (GMA)}
	\begin{examplesecond}{Définition}
	
		Un marquage sera dit accessible si on peut l'atteindre à partir du marquage initial, soit directement (avec un seul tir), soit indirectement (avec plusieurs tirs).
		On note A l'ensemble des marquages accessibles d'un réseau de Petri.
	\end{examplesecond}
	
	\begin{center}
				
		\begin{minipage}[h]{0.2\linewidth}
			\includegraphics<1>[scale=0.4]{GMA_exemple.png}
		\end{minipage}
		\hspace{0.5cm}	
		\begin{minipage}[h]{0.58\linewidth}
			Exemple (précédent) : \\
			A = M0 , M1 , M2\\
			Le graphe des marquages accessibles:
		\end{minipage}

	\end{center}
	
	\end{frame}
	
	
	\begin{frame}
		\frametitle{Graphe de Marquage Accessible (GMA)}
		\textbf{Application}
		Considérons le réseau de Petri suivant :
		
		\centering
		\includegraphics<1>[scale=0.3]{application1.png}
			
			
			soit T = ($t_{1},t_{2}$), on a:\\
			$C^{-}$ = Pré = 
			\begin{pmatrix}
			1 & 0 \\
			0 & 1 \\
			0 & 1 \\
			1 & 0
			\end{pmatrix};
			$C^{+}$ = Post = 
			\begin{pmatrix}
			1 & 0 \\
			0 & 1 \\
			1 & 0 \\
			0 & 1
			\end{pmatrix};\\
			C = $C^{+}$ - $C^{-}$ = 
			\begin{pmatrix}
			0 & 0 \\
			0 & 0 \\
			1 & -1 \\
			-1 & 1
			\end{pmatrix}
		
		
	\end{frame}
	
	\begin{frame}
		\frametitle{Graphe de Marquage Accessible (GMA)}
		
		\textbf{Application}(suite)

		
	\begin{minipage}[h]{0.2\linewidth}
		\includegraphics<1>[scale=0.4]{application2.png}
	\end{minipage}
	\hspace{1cm}
	\begin{minipage}[h]{0.6\linewidth}
		avec M0 = (1, 1, 0, 4)\\
		On peut avoir les autres marquages accessibles,\\ exemple :
		M0 [t1 >M1 car [1, 1, 0, 4] > [1, 0, 0, 1]\\
		
		\newline
		
		Donc : M1 = M0 +1 C = [1, 1, 0, 4] + [0, 0, 1, 3] etc.\\
		Le graphe des marquages accessibles :
	\end{minipage}
	\end{frame}
	
	
	
	\subsection{ Le vecteur d’occurrence et l’équation de changement d’état }
		\begin{frame}
	\frametitle{Le vecteur d’occurrence et l’équation de changement d’état}
	\begin{examplesecond}

		Soit $T^{*}$ l’ensemble de transitions $T^{*}$ = t1 , t2 , . . . , tn\\
		Soit S une séquence de transitions (S ∈ $T^{*}$ ) ; $\overrightarrow{\sigma} = (\overrightarrow{\sigma} (t))_{t}$\\
		où $\overrightarrow{\sigma}$(t) est le nombre d’occurrences de t dans S.	
	\end{examplesecond}
	
	\textbf{Exemple 1 :}\\
	 Soit le graphe de marquage ci-dessous,\\
	
	\begin{minipage}[h]{0.2\linewidth}
	 	\includegraphics<1>[scale=0.4]{exemple_page9.png}
	\end{minipage}
	\hspace{1.5cm}
	\begin{minipage}[h]{0.6\linewidth}
		avec T = t1 , t2 , t3.\\
		Considérons la séquence $\overrightarrow{\sigma}$ = t1 t2 t3\\
		Alors le vecteur d'occurrences $\overrightarrow{\sigma} = (\overrightarrow{\sigma} (t1 ), \overrightarrow{\sigma} (t2 ), \overrightarrow{\sigma} (t3 )) est \overrightarrow{\sigma}$\\
		= (2, 1, 0)\\
	\end{minipage}

	

	\end{frame}
	
	\begin{frame}
		\frametitle{Le vecteur d’occurrence et l’équation de changement d’état}
		
			Ainsi, à partir d'un marquage M , on peut tirer une séquence de transitions $\sigma$, et on trouve
			le marquage $M^{'}$ .\\
			
		\begin{examplesecond}
			
			\centering
			L'équation de changement d'état est alors donnée comme suit :\\
			
			\textbf{$M_{0} = M + C . \overrightarrow{\sigma}$}
		\end{examplesecond}
		
	\end{frame}
	
	\begin{frame}
		\frametitle{Le vecteur d’occurrence et l’équation de changement d’état}
		
		\textbf{Exemple 2 :}
		le franchissement d'une transition «source» consiste à rajouter un jeton à chacune
		des places en sortie.\\
		
		\begin{center}
			\includegraphics<1>[scale=0.4]{exemple_page10_1.png}
		\end{center}
	\end{frame}
	
	\begin{frame}
		\frametitle{Le vecteur d’occurrence et l’équation de changement d’état}
		
		\textbf{Exemple 3 :}
		le franchissement d'une transition «puits» consiste à retirer un jeton de chacune
		de ses places en entrée.\\
		
		\begin{center}
			\includegraphics<1>[scale=0.4]{exemple_page10_2.png}
		\end{center}
	\end{frame}
	
	
		\begin{frame}
			\frametitle{Le vecteur d’occurrence et l’équation de changement d’état}
			
			\textbf{Exemple 4 :}\\
				séquence de franchissement :\\
				M0 [t1 >M1 avec M1 = (0, 1, 0, 0)\\
				M0 [t1 t2 >M2 avec M2 = (0, 0, 0, 1)\\
				M0 [t1 t3 >M3 avec M3 = (0, 0, 0, 1)\\
				Ensemble des marquages accessibles : M ∗ = M0 , M1 , M1 , M3\\
			
			\begin{center}
				\includegraphics<1>[scale=0.35]{exemple_page10_3.png}
			\end{center}
		\end{frame}

	
	\subsection{ Quelques propriétés qualitatives                 }

	\subsubsection{ Bornitude                         }
		\begin{frame}
	\frametitle{Quelques propriétés qualitatives}
	
	Quelques propriétés qualitatives Vérifiables en se basant sur le graphe des marquages accessibles.\\
	\\
	\textbf{  }\\
	
	\textbf{Bornitude}
	\begin{examplesecond}{Définition}
		Une place sera dite k-bornée si ∀M, M (p) ≤ k
	\end{examplesecond}
	
	Exemple (précédent):
	– $M_{0}$(p1 ) = $M_{1}$(p1 ) = $M_{2}$(p2 ) = $M_{3}$(p3 ) = $M_{4}$(p4 ) = 1 donc p1 est 1-bornée
	– la place p4 est 4-bornée
	Un réseau de Petri est borné s'il existe une valeur k telle que :
	∀M, ∀p, M (p) ≤ k
	\end{frame}
	\subsubsection{ Pseudo-vivacité                      }
		\begin{frame}
	\frametitle{}
	\end{frame}
	\subsubsection{ Quasi-vivacité                       }
		\begin{frame}
	\frametitle{}
	\end{frame}
	\subsubsection{ Vivacté                          }
		\begin{frame}
	\frametitle{}
	\end{frame}
	\subsubsection{ Réseau sans blocage                   }
		\begin{frame}
	\frametitle{}
	\end{frame}
	\subsubsection{ Etat d’accueil                       }
		\begin{frame}
	\frametitle{}
	\end{frame}
	\subsubsection{ Conservation                       }
		\begin{frame}
	\frametitle{}
	\end{frame}
	\subsection{ Types de réseaux de Petri                    }
		\begin{frame}
	\frametitle{}
	\end{frame}
	\subsubsection{ Réseaux de Petri généralisés               }
		\begin{frame}
	\frametitle{}
	\end{frame}
	\subsubsection{ Réseaux de Petri à capacités              }
		\begin{frame}
	\frametitle{}
	\end{frame}
	\subsubsection{ Graphe de marquage                   }
		\begin{frame}
	\frametitle{}
	\end{frame}
	\subsection{ Arborescence de couverture                   }
		\begin{frame}
	\frametitle{}
	\end{frame}
	\subsubsection{ Algorithme de contstruction d’un graphe de marquage}
		\begin{frame}
	\frametitle{}
	\end{frame}
\end{document}